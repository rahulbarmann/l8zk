\noindent\textbf{Google's longfellow.}~\cite{cryptoeprint:2024/2010} This approach targets ecosystems where issuers already sign credentials using ECDSA on standard 
curves (e.g., P\textendash256) and hash data with SHA\textendash256. The core difficulty is that proving 
ECDSA verification in zero knowledge is expensive for conventional proof systems: the arithmetic of 
P\textendash256 and the bitwise structure of SHA\textendash256 do not align with the fast polynomial 
techniques (such as NTT-based optimizations) used in many modern ZK libraries.

To address this, the authors design custom circuits for both ECDSA and SHA\textendash256 and employ a 
layered protocol built on the sum-check technique with a lightweight Reed--Solomon encoding to keep 
proof sizes manageable. A consistency check ensures that the same hidden signing key is used in both 
the signature and hashing logic. During presentation, the wallet generates a fresh proof and the 
device signs the verifier’s challenge, providing device binding.

In the reference-system perspective, issuer compatibility is fully preserved, selective disclosure is 
supported, and device binding is incorporated. The drawback is the absence of a reusable offline 
phase, meaning each presentation requires generating a full proof. Reported performance is 
approximately 60\,ms to prove a single ECDSA signature and around 1.2\,s for a complete mDL 
presentation on mobile devices~\cite[\S5.3,\S6.2]{cryptoeprint:2024/2010}, with larger proofs and 
higher verification overhead than succinct SNARK-based systems with trusted setup.
