%ChatGPT except the footnote and the box

\noindent\textbf{Anonymous Credentials from Existing Digital Identity.} We focus on constructing anonymous credentials for 
\emph{existing} digital identity systems without requiring changes to the current issuance flow whilst 
enabling selectively disclosed presentations in an unlinkable manner. 

A central target of this work is the \emph{EUDI ARF}, which specifies how digital identities and credentials are issued, stored, and presented within the EUDI ARF.
While our terminology in this section follows the generic issuer–holder–verifier model, Section~\ref{subsec:aligneudi} restates the same roles using the Wallet User, Relying Party, and EAA vocabulary of the EUDI ARF.
Our construction is designed to integrate with this framework in a modular  manner, allowing anonymous credentials to be derived from standard verifiable credentials, 
whilst complying with its security, interoperability, and regulatory requirements.\footnote{The EU Digital Identity Wallet will be launched in all member states by the end of 2026, with strong requirements for unlinkability with respect to relying parties and identity providers.
According to the Cryptographers' Feedback on the EU Digital Identity’s ARF\footnote{\url{https://github.com/user-attachments/files/15904122/cryptographers-feedback.pdf}}, an Anonymous Credential AC scheme is a suitable cryptographic primitive to instantiate the new EU Digital Identity Wallet (EUDIW, ARF version 1.4.0~\cite{EU:EUDI24}), which is an important step towards developing interoperable digital identities in Europe for the public and private sectors.}

\begin{framed}\footnotesize
	Informally speaking, an \emph{Anonymous Credential} AC scheme allows:
	\begin{itemize}
		\item An \emph{Identity Provider} or \emph{Issuer} IP to (possibly blindly\footnote{i.e. the IP does not know the content that it signs, only its provenance is satisfied.}) sign a set of (eligible) attributes for a \emph{User} U;
		\item The \emph{User} U can show, only if they hold the signed attributes (a.k.a \emph{Unforgeability}), usually through a \emph{Presentation}, to a \emph{Relying Party} RP such that:
		\begin{itemize}
			\item The RP can verify that the set of attributes (signed by IP) that the User U holds satisfy some condition of their interest (a.k.a Correctness);
			\item The RP cannot learn any \emph{additional}\footnote{We stress that the RP may have obtained some privacy sensitive information prior to this presentation.} information beyond the fact that the condition is satisfied or information that can be inferred from the satisfaction of the condition (a.k.a Zero-Knowledge or Anonymity);
			\item The immediate previous requirement also implies that the RP cannot link the various presentations by the same User U (a.k.a. Unlinkability);
		\end{itemize}
		\item The IP can revoke all or a part of the signed attributes that it has issued to the User U, from upon which, the eligible attributes of the User U are updated, and subsequent presentations have to be based on the new and updated attributes (a.k.a \emph{Revocation});
		\item The User U cannot transfer its set of signed attributes to another User U' (a.k.a \emph{Non-transferability}).
	\end{itemize}
\end{framed}