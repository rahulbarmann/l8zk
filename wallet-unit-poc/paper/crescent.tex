\noindent\textbf{Microsoft's Crescent}~\cite{cryptoeprint:2024/2013} targets environments in which issuers keep their existing credential formats (e.g., JWT, 
mDL) and continue using their current signing keys, requiring no issuer-side changes. Its workflow 
splits into a heavy one-time \emph{Prepare} phase and a lightweight per-session \emph{Show} phase. 
During Prepare, the wallet verifies the issuer’s signature, parses attributes, and constructs two 
reusable artifacts: (i) a Groth16 proof attesting correct verification and parsing, and (ii) a 
Pedersen vector commitment enabling selective disclosure. Both artifacts can be re-randomized to 
ensure unlinkability.

In the Show phase, the wallet re-randomizes these artifacts and includes only the proofs required by 
the verifier’s policy, such as age checks or linking two credentials to the same holder. Device 
binding can be added by having the secure element sign the verifier’s challenge. In reference-system 
terms, Crescent achieves a two-phase model with reusable offline work and modular predicates while 
leaving issuers untouched.

The trade-offs are substantial: the Prepare phase is expensive (tens of seconds for JWTs and minutes 
for mDLs), the system relies on pairing-based Groth16 proofs with a large trusted setup 
($\approx$661\,MB--1.1\,GB~\cite[\S4]{cryptoeprint:2024/2013}), and the security guarantees are 
classical only. The Show phase is fast—typically 22--41\,ms with $\approx$1\,KB proofs, or around 
315\,ms when device binding is included~\cite[\S4]{cryptoeprint:2024/2013}.
