\noindent\textbf{A Generic AC Framework.} In a generic AC framework, the \emph{Issuer} is an entity that uses a fixed public-key 
algorithm (e.g., RSA or ECDSA) and verifiable credential format (such as SD-JWT or mDL) to assert claims about one or more subjects. Since issuer elements are 
typically difficult to modify once deployed, the proposed change are to the holder’s 
wallet and to the relying party (verifier).

The wallet operates in two phases. 
During an offline \emph{Prepare} phase, run once per credential, the wallet:
 \begin{enumerate}
    \item Verifies the Issuer’s signature using standard libraries
    \item Parses and normalizes credential attributes (e.g., converting a 
date of birth to an integer age)
    \item Commits to the attributes using a binding and hiding commitment scheme. 
\end{enumerate}
During the online \emph{Show} phase, which is run for each presentation, the wallet:
\begin{enumerate}
\item Selects the attributes or predicates required by the \emph{Relying Party}
\item Prove them in zero knowledge against the stored commitments
\item Adds a fresh device signature over the session challenge to ensure device binding.
\end{enumerate}

Noticeably, a key requirement of our framework is \emph{modularity}: issuer-signature verification, attribute commitment, predicate proofs, and device binding are defined as separate modules with clear interfaces.
This separation allows the underlying proof system to be replaced for quantum-proofing without modifying other components. 
