%ChatGPT

\noindent\textbf{Digital Identity.} A digital identity system typically consists of three roles: an \emph{issuer}, a \emph{holder}, and a 
\emph{verifier}. In the W3C Verifiable Credential Data Model, the issuer is a trusted authority responsible for  asserting \href{https://www.w3.org/TR/vc-data-model-1.1/\#dfn-claims}{claims} about one or more \href{https://www.w3.org/TR/vc-data-model-1.1/\#dfn-subjects}{subjects}, and creating a \href{https://www.w3.org/TR/vc-data-model-1.1/\#dfn-verifiable-credentials}{verifiable credential} from these \href{https://www.w3.org/TR/vc-data-model-1.1/\#dfn-claims}{claims}. The claims are digitally 
signed with the issuer’s private key. The signed credential is delivered securely to the user and stored in the wallet instance.

The holder retains these signed verifiable credentials and, during a presentation request, provides consent based on the verifier’s requirements. The holder sends the verifiable presentation (or proofs derived from them) to the verifier. The verifier, using the issuer’s public 
key, checks the digital signature to confirm authenticity and integrity of the received presentation, 
and may additionally verify validity period, or revocation status. This issuer-holder-
verifier pattern defines the standard operational model.

Such a system is illustrated in Fig.~\ref{fig:did_sys}.
% \begin{figure}[!h]
%     \centering
%     \resizebox{15cm}{!}{
%         \begin{tikzpicture}[
%           entity/.style = {draw, thick, rounded corners, minimum width=3.2cm, minimum height=1.2cm, align=center, fill=white},
%           edge/.style   = {->, >=Stealth, thick},
%           dashededge/.style = {->, >=Stealth, thick, dashed},
%           smalltext/.style = {font=\small, align=center}
%           ]
        
%           % Entities
%           \node[entity] (issuer) {Issuer\\(e.g., government, bank)};
%           \node[entity, right=4.8cm of issuer] (holder) {Holder / User\\(stores credential)};
%           \node[entity, right=4.8cm of holder] (verifier) {Verifier\\(relying party)};
        
%           % Issuance arrow: Issuer -> Holder
%           \draw[edge] (issuer.east) to[bend left=12] node[midway, above, smalltext] {Signed attribute set\\(credential)} (holder.west);
        
%           % Presentation arrow: Holder -> Verifier
%           \draw[edge] (holder.east) to[bend left=12] node[midway, above, smalltext] {Select attributes or proof\\presented to verifier} (verifier.west);
        
%           % Verification arrow: Verifier checks signature using issuer public key
%           \draw[edge] (verifier.west) to[bend left=25] node[midway, below, smalltext] {Verify signature \\(using issuer pk)} (issuer.east);
        
%           % Local storage box around holder to indicate secure storage (e.g., mobile wallet)
%           \node[draw=black, rounded corners, inner sep=6pt, fit=(holder), label=below:Secure storage (wallet)] {};
        
%           % Optional: Revocation / freshness check
%           \node[entity, below=3.0cm of holder] (revocation) {Revocation / Status Service};
%           \draw[dashededge] (verifier.south) to[bend left=12] node[midway, right, smalltext] {Check revocation \\/ freshness} (revocation.north);
%           \draw[dashededge] (revocation.north) to[bend left=12] node[midway, right, smalltext] {Respond\\ status} (verifier.south);
        
%           % Notes
%           \node[align=left, font=\footnotesize, below=0.6cm of revocation] {
%             \textbf{Notes:}\\
%             \begin{minipage}{20cm}
%               \vspace{1mm}
%               \begin{itemize}\itemsep1pt
%                 \item The issuer signs a set of user attributes and delivers the credential to the holder.
%                 \item The holder stores the signed attributes and selectively presents either the raw attributes or a derived proof to the verifier.
%                 \item The verifier validates authenticity by checking the issuer's signature and may consult a revocation/status service.
%               \end{itemize}
%             \end{minipage}
%           };
        
%         \end{tikzpicture}
%     }
%     \caption{Digital identity system}
%     \label{fig:did_sys}
% \end{figure}
\begin{figure}[!h]
  \centering
  \resizebox{0.9\linewidth}{!}{%
  \begin{tikzpicture}[
    entity/.style = {
      draw, thick, rounded corners,
      minimum height=1.1cm,
      align=center,
      inner sep=4pt,
      fill=white
    },
    edge/.style      = {->, >=stealth, thick, rounded corners=8pt},
    smalltext/.style = {font=\small, align=center}
  ]

    \node[entity, text width=3.8cm] (issuer)
      {Issuer\\\footnotesize(issues VCs)};
    \node[entity, right=4.8cm of issuer, text width=4.2cm] (holder)
      {Holder / User\\\footnotesize(stores VCs, creates VPs)};
    \node[entity, right=4.8cm of holder, text width=3.8cm] (verifier)
      {Verifier\\\footnotesize(verifies VPs)};

    \node[draw=black, rounded corners, inner sep=6pt,
          fit=(holder),
          label=above:Secure storage (wallet)] {};

    \node[entity, below=3.0cm of holder, text width=8.0cm] (registry)
      {Verifiable Data Registry\\
       \footnotesize(identifiers, schemas)};

    \draw[edge]
      (issuer.east) --
      node[pos=0.5, above, smalltext] {Issue credential (VC)}
      (holder.west);

    \draw[edge]
      (holder.east) --
      node[pos=0.5, above, smalltext] {Presentation (VP)}
      (verifier.west);

    \draw[edge]
      (issuer.south) |- 
      node[pos=0.25, left, smalltext]
        {Publish schemas}
      (registry.west);

    \draw[edge]
      (holder.south) --
      node[midway, right, smalltext]
        {Register identifiers\\and use schemas}
      (registry.north);

    \draw[edge]
      (verifier.south) |- 
      node[pos=0.25, right, smalltext]
        {Query\\identifiers \& schemas}
      (registry.east);

    % Notes
    \node[align=left, font=\footnotesize, below=0.6cm of registry] {
      \textbf{Notes:}\\
      \begin{minipage}{20cm}
        \vspace{1mm}
        \begin{itemize}\itemsep1pt
          \item The issuer issues a verifiable credential (VC) containing a set of user attributes and delivers it to the holder.
          \item The holder stores VCs in a secure wallet and selectively discloses attributes or derived proofs as a verifiable presentation (VP) to the verifier.
          \item The verifier validates authenticity by checking the issuer's signature using the issuer's public key (resolved via the verifiable data registry) and consults the registry to resolve identifiers and schemas.
        \end{itemize}
      \end{minipage}
    };

  \end{tikzpicture}%
  }
  \caption{Digital identity system}
  \label{fig:did_sys}
\end{figure}


Examples of such digital identity frameworks and credential formats that can be potentially compatible with OpenAC include:
\begin{itemize}
  \item \textbf{eIDAS / EUDI Architecture and Reference Framework} -European digital identity systems in which 
  qualified trust service providers or public authorities issue digitally signed identity 
  attributes that citizens can present via the EUDI Wallet.

  \item \textbf{Mobile Driver’s Licenses (mDL, ISO~18013-5)} - Government agencies issue a 
  digitally signed set of driving-related attributes (e.g., name, age, license class), which 
  users present to relying parties such as law enforcement or age-restricted service providers.

  \item \textbf{Verifiable Credentials (W3C VCDM)} -A general-purpose model where an issuer 
  signs JSON-LD or SD-JWT-based claims about a subject; holders store these credentials and 
  present them to verifiers who validate the issuer’s signature.

  % \item \textbf{Bank-Issued Digital Identity Schemes} --- Financial institutions validate 
  % customer identity information (e.g., KYC data) and issue signed digital credentials that can 
  % be reused for onboarding in other financial or commercial services.

  % \item \textbf{University or Organizational ID Systems} --- Institutions issue digitally 
  % signed attestations (e.g., enrollment status, roles, or qualifications) that students or 
  % employees can present to external verifiers for access to services or discounts.

  % \item \textbf{National Digital ID Cards (e.g., Estonia eID)} --- Government-issued smartcards 
  % or mobile IDs contain signed identity attributes that citizens present for authentication and 
  % authorization in public and private sector services.
\end{itemize}


%ChatGPT except the footnote and the box

\noindent\textbf{Anonymous Credentials from Existing Digital Identity.} We focus on constructing anonymous credentials for 
\emph{existing} digital identity systems without requiring changes to the current issuance flow whilst 
enabling selectively disclosed presentations in an unlinkable manner. 

A central target of this work is the \emph{EUDI ARF}, which specifies how digital identities and credentials are issued, stored, and presented within the EUDI ARF.
While our terminology in this section follows the generic issuer–holder–verifier model, Section~\ref{subsec:aligneudi} restates the same roles using the Wallet User, Relying Party, and EAA vocabulary of the EUDI ARF.
Our construction is designed to integrate with this framework in a modular  manner, allowing anonymous credentials to be derived from standard verifiable credentials, 
whilst complying with its security, interoperability, and regulatory requirements.\footnote{The EU Digital Identity Wallet will be launched in all member states by the end of 2026, with strong requirements for unlinkability with respect to relying parties and identity providers.
According to the Cryptographers' Feedback on the EU Digital Identity’s ARF\footnote{\url{https://github.com/user-attachments/files/15904122/cryptographers-feedback.pdf}}, an Anonymous Credential AC scheme is a suitable cryptographic primitive to instantiate the new EU Digital Identity Wallet (EUDIW, ARF version 1.4.0~\cite{EU:EUDI24}), which is an important step towards developing interoperable digital identities in Europe for the public and private sectors.}

\begin{framed}\footnotesize
	Informally speaking, an \emph{Anonymous Credential} AC scheme allows:
	\begin{itemize}
		\item An \emph{Identity Provider} or \emph{Issuer} IP to (possibly blindly\footnote{i.e. the IP does not know the content that it signs, only its provenance is satisfied.}) sign a set of (eligible) attributes for a \emph{User} U;
		\item The \emph{User} U can show, only if they hold the signed attributes (a.k.a \emph{Unforgeability}), usually through a \emph{Presentation}, to a \emph{Relying Party} RP such that:
		\begin{itemize}
			\item The RP can verify that the set of attributes (signed by IP) that the User U holds satisfy some condition of their interest (a.k.a Correctness);
			\item The RP cannot learn any \emph{additional}\footnote{We stress that the RP may have obtained some privacy sensitive information prior to this presentation.} information beyond the fact that the condition is satisfied or information that can be inferred from the satisfaction of the condition (a.k.a Zero-Knowledge or Anonymity);
			\item The immediate previous requirement also implies that the RP cannot link the various presentations by the same User U (a.k.a. Unlinkability);
		\end{itemize}
		\item The IP can revoke all or a part of the signed attributes that it has issued to the User U, from upon which, the eligible attributes of the User U are updated, and subsequent presentations have to be based on the new and updated attributes (a.k.a \emph{Revocation});
		\item The User U cannot transfer its set of signed attributes to another User U' (a.k.a \emph{Non-transferability}).
	\end{itemize}
\end{framed}

\noindent\textbf{A Generic AC Framework.} In a generic AC framework, the \emph{Issuer} is an entity that uses a fixed public-key 
algorithm (e.g., RSA or ECDSA) and verifiable credential format (such as SD-JWT or mDL) to assert claims about one or more subjects. Since issuer elements are 
typically difficult to modify once deployed, the proposed change are to the holder’s 
wallet and to the relying party (verifier).

The wallet operates in two phases. 
During an offline \emph{Prepare} phase, run once per credential, the wallet:
 \begin{enumerate}
    \item Verifies the Issuer’s signature using standard libraries
    \item Parses and normalizes credential attributes (e.g., converting a 
date of birth to an integer age)
    \item Commits to the attributes using a binding and hiding commitment scheme. 
\end{enumerate}
During the online \emph{Show} phase, which is run for each presentation, the wallet:
\begin{enumerate}
\item Selects the attributes or predicates required by the \emph{Relying Party}
\item Prove them in zero knowledge against the stored commitments
\item Adds a fresh device signature over the session challenge to ensure device binding.
\end{enumerate}

Noticeably, a key requirement of our framework is \emph{modularity}: issuer-signature verification, attribute commitment, predicate proofs, and device binding are defined as separate modules with clear interfaces.
This separation allows the underlying proof system to be replaced for quantum-proofing without modifying other components. 


%ChatGPT, some mods
\medskip
\noindent\textbf{Related Work and Our Proposal.} There exists several prominent approaches to constructing anonymous credentials 
from existing digital identity systems, including the BBS$+$ proposal~\cite{SCN:AuSusMu06,EPRINT:CamDriLeh16}, Microsoft’s \emph{Crescent}~\cite{cryptoeprint:2024/2013}, and Google’s \emph{Longfellow}~\cite{cryptoeprint:2024/2010}. 
Although each of these schemes enables some form of unlinkable selective disclosure, they exhibit limitations in practicality or generality that our design aims to overcome.

The BBS$+$ approach requires modifications to the issuer’s existing credential-issuance flow, 
thereby reducing compatibility with established digital identity infrastructures. The Crescent system 
relies on a trusted setup, which introduces transparency concerns and operational overhead for 
ecosystem deployment. The Longfellow construction, while efficient, is tailored specifically to 
ECDSA-based identity systems and thus lacks general applicability across heterogeneous credential formats.

Our proposal offers an \emph{open} and \emph{fully transparent} design that imposes 
no changes on the issuer workflow, supports modular ``plug-and-play’’ integration with diverse 
credential systems, and requires no trusted setup. Furthermore, our proof-of-concept implementation 
demonstrates best-in-class proving performance, including efficient execution on mobile devices, 
which is essential for real-world usability in frameworks such as the EUDI Wallet.

%Below is by YT
A summary of the existing solutions and our proposal with respect to the AC Framework is presented in Table~\ref{tab:rw-snapshot}. We additionally adapt a comparison for a 1920-byte credential from Vega~\cite{kaviani2025vega}, a concurrent work that follows the similar prepare-and-prove paradigm.\footnote{We have not yet attempted to compare our work with Vega in this version, we will conduct a more detailed analysis of Vega in the next version.}
For OpenAC, the four latency columns follow the same convention as Vega: \emph{Setup} covers the proof system setup for the Prepare and Show circuits; \emph{Precomp.} is the one-time offline Prepare work; \emph{Prove} is the online Show work for a single presentation; and \emph{Verify} is the verifier’s work across both phases.
The proof size column reports the sum of the Prepare and Show proof sizes, and the PK/VK columns give the total proving and verifying key sizes for the two circuits.

We consider a client–server setting with a mobile prover and a server-side verifier.
The verifying key, which is dominated by the Prepare circuit, is large but kept in memory on the verifier side and reused across many proof verifications, while the prover runs under tighter latency and resource constraints.
For consistency with Vega, Table~\ref{tab:openac-main} reports measurements on commodity desktop hardware; mobile measurements and a more detailed benchmarking discussion, including a breakdown from the wallet’s perspective, appear in Section~\ref{sec:benchmarks}.

\begin{table}[h]
\centering
\label{tab:rw-snapshot}
\caption{Comparison of related approaches.}
\footnotesize
\begin{tabularx}{\linewidth}{
  @{}
  >{\RaggedRight\arraybackslash}p{0.17\linewidth}  
  >{\RaggedRight\arraybackslash}p{0.14\linewidth}  
  >{\RaggedRight\arraybackslash}p{0.16\linewidth}  
  >{\RaggedRight\arraybackslash}p{0.21\linewidth}  
  >{\RaggedRight\arraybackslash}p{0.19\linewidth}  
  @{}
}
\toprule
\textbf{Feature} & \textbf{BBS/BBS+} & \textbf{Longfellow} & \textbf{Crescent} & \textbf{OpenAC} \\
\midrule
Issuer modification & Required & None & None & None \\
\addlinespace[0.3em]
Offline phase       & None & None & Lightweight, reusable Prepare & Lightweight, reusable Prepare \\
\addlinespace[0.3em]
Setup               & Pairing-based (no trusted setup) & Transparent & Large, per-circuit trusted setup & Transparent \\
\addlinespace[0.3em]
Proof mechanism     & Pairing-based signatures + ZK proofs & Sum-check + Ligero; custom ECDSA/SHA-256 circuits & Groth16 + Pedersen vector commitments; re-randomizable artifacts & Sum-check; Hyrax-style vector commitments \\
\addlinespace[0.3em]
Device binding      & Optional & Included & Optional & Integrated (in-circuit) \\
\addlinespace[0.3em]
Reusability         & No & No & Yes & Yes \\
\bottomrule
\end{tabularx}

\end{table}

% \begin{table}[h!]
% \centering
% \renewcommand{\arraystretch}{1.2}
% \setlength{\tabcolsep}{6pt}

% \begin{tabular}{l|cccc|ccc|>{\centering\arraybackslash}p{1.3cm}} 
% \hline
% \multirow{2}{*}{\textbf{Scheme}} &
% \multicolumn{4}{c|}{\textbf{Latency (ms)}} &
% \multicolumn{3}{c|}{\textbf{Size (kB)}} &
% \multirow{2}{*}{\textbf{\shortstack{Trans.\\Setup}}} \\  
% & \textit{Setup} & \textit{Precompute} & \textit{Prove} & \textit{Verify} &
%   \textit{Proof} & \textit{pk} & \textit{vk} \\
% \hline
% Longfellow & 7235 & ---    & 680 & 324 & 325 & 202     & 202   & \checkmark \\
% Crescent   & 172\,437 & 14\,725 & 237 & 118 & 16 & 710\,565 & 1 & $\times$ \\
% Vega\textsubscript{SC} & 3689 & 238 & 247 & 55 & 99 & 6562  & 6561 & \checkmark \\
% Vega\textsubscript{MC} & 193  & 109 & 212 & 51 & 150 & 436   & 436  & \checkmark \\
% OpenAC  &  5230   & 3016    &     &     &     &      &      & \checkmark \\
% \hline
% \end{tabular}

% \caption{Performance comparison for a 1920-byte MSO (latency in ms, sizes in kB), adapted from Vega~\cite{kaviani2025vega}}
% \label{tab:rw-perfcomp}
% \end{table}


\begin{table}[t]
\centering
\caption{Performance comparison for a 1920-byte MSO, adapted from Vega~\cite{kaviani2025vega}.}
\scriptsize
\setlength{\tabcolsep}{3pt}
\renewcommand{\arraystretch}{1.1}
\begin{tabular}{l|cccc|ccc|c} 
\hline
\multirow{2}{*}{\textbf{Scheme}} &
\multicolumn{4}{c|}{\textbf{Latency (ms)}} &
\multicolumn{3}{c|}{\textbf{Size (kB)}} &
\multirow{2}{*}{\textbf{\shortstack{Trans.\\Setup}}} \\ 
& \textit{Setup} & \textit{Precomp.} & \textit{Prove} & \textit{Verify} &
  \textit{Proof} & \textit{pk} & \textit{vk} \\
\hline
Longfellow      & 7\,235   & ---    & 680 & 324 & 325 & 202     & 202   & \checkmark \\
Crescent        & 172\,437 & 14\,725 & 237 & 118 & 16  & 710\,565 & 1   & $\times$ \\
Vega\textsubscript{SC} & 3\,689   & 238    & 247 & 55  & 99  & 6\,562  & 6\,561 & \checkmark \\
Vega\textsubscript{MC} & 193     & 109    & 212 & 51  & 150 & 436     & 436   & \checkmark \\
\hline
OpenAC & 
4193 &    % Setup (Prepare + Show setup) 4157 + 36
3442 &    % Precomp (Prepare Prove + Reblind Prepare) 2727 + 715
102 &     % Prove (Show Prove + Reblind Show) 77 + 25
83 &    % Verify (Prepare Verify + Show Verify) 74 + 9
149.7 &   % Proof size (Prepare proof + Show proof) 109.29 + 40.41
433664 &  % Proving key size: Prepare Proving Key + Show Proving Key 420.05MB + 3.45MB
433664 &  % Verifying key size: Prepare Verifing Key + Show Verifying Key
\checkmark \\
\hline
\end{tabular}
\vspace*{0.3cm}
\begin{minipage}[t]{0.8\textwidth}
    The Longfellow, Crescent, and Vega benchmarks are done in Azure Standard\_F16as\_v6 VM with 16 vCPUs and 64 GB RAM while we consider a commodity hardware (MacBook Pro, M4, 14-core GPU, 24GB RAM).
\end{minipage}
\label{tab:openac-main}
\end{table}



% In the aforementioned feedback document, BBS~\cite{C:BonBoySha04, C:CamLys04} and BBS+~\cite{SCN:AuSusMu06,EPRINT:CamDriLeh16}
% %\footnote{For BBS, thanks to prior work by the W3C, the Decentralized Identity Foundation, IETF/IRTF, ISO, and other standardization bodies, as well as the availability of open-source software libraries, the EC can develop a standard and reference implementation with only a modest effort. The feedback additionally recommend that the EUDI be designed following the principle of crypto-agility, meaning that its underlying technologies can be upgraded quickly in the future if the need arises.} 
% were promoted as the main candidate, besides that, there have been two independent works from Google~\cite{cryptoeprint:2024/2010} (longfellow) and Microsoft~\cite{cryptoeprint:2024/2013} (Crescent) that attempted to offer candidate solutions. In this document, we attempt to offer a new candidate, called \textbf{zkID}.

% In comparison, these approaches show the current trade-off: systems either reuse existing issuer infrastructure but pay high per-presentation costs, or they achieve fast online proofs at the price of large setups and pairing-based assumptions. 
% \begin{quote}
% 	\emph{The zkID construction aims to combine issuer compatibility with reusable offline work, while remaining transparent and modular.}
% \end{quote}

% \paragraph{Organization}
% The remainder of this document is organized as follows: Section~\ref{sec:appeudi} introduces the EUDI setting and maps its requirements to our construction, Section~\ref{sec:preliminaries} introduces the notation and cryptographic tools, Section~\ref{sec:contribution} describes the technical details of our zkID construction, Section~\ref{sec:security} provides the security analysis and proofs, Section~\ref{sec:benchmarks} reports benchmark results, and Section~\ref{sec:conclusion} concludes.

% \subsection{Our zkID}
% We work with SD--JWT credentials, extensions to other formats are straightforward. The construction adds a zero-knowledge layer around the issued credential and is instantiated with Spartan as the proof system and Hyrax-style Pedersen vector commitments; standard zero-knowledge blinding is applied.

% \paragraph{Objects and notation.}
% Let $S$ be the issued credential with messages $(m_1,\dots,m_N)$, per-message salts $(s_1,\dots,s_N)$, hashes $(h_1,\dots,h_N)$, and issuer signature $\sigma_I$ under public key $PK_I$. The wallet public key $PK_W$ is included among the messages. A presentation policy is modeled as predicates $f_1,\dots,f_K$ over $(m_i)$.

% \paragraph{Two relations.}
% Proving is split into two relations with a fixed interface:
% \begin{itemize}[leftmargin=1.2em]
%   \item Prepare: Once per credential, prove the correct parsing of $S$, that $h_i=\mathrm{SHA256}(m_i,s_i)$ for all $i$, that $\mathrm{Verify}(\sigma_I,PK_I)=1$, and compute a Hyrax commitment $C_m$ to the \emph{message vector} $(m_i)$. This relation is independent of the presentation policy; its proof can be re-randomized and reused.
%   \item Show: For a given policy, prove that each requested predicate $f_j(m_1,\dots,m_N)$ holds and that a live signature on the verifier’s challenge verifies under the wallet public key bound in $S$. Publish a commitment $C'_m$ to the same message vector.
% \end{itemize}
% The link is enforced by checking $C_m=C'_m$ (Hyrax commitment equality), so no auxiliary link primitive is required. Because equality is checked in one group, both relations use the same curve.

% \paragraph{Workflow.}
% Issuance produces $S$ and binds it to $PK_W$. At presentation, the verifier sends a fresh challenge $\mathsf{nonce}_V$; the wallet signs $\mathsf{nonce}_V$ with the corresponding secret key and returns $(\pi_{\texttt{prepare}},\pi_{\texttt{show}})$. The verifier validates both proofs and accepts only if $C_m=C'_m$.

% \subsection{Comparison to alternatives}\label{subsec:comparison}

% \paragraph{Setting the AC Framework.}
% Let us first outline a reference architecture that represents what an anonymous-credential system would ideally look like if it is to integrate smoothly with current infrastructures. 
% In this model, the \emph{Issuer} is treated as a fixed component that continues to use its existing public-key algorithms (such as RSA or ECDSA) and standard credential formats (e.g., JWT or mDL), since it's typically difficult to change once deployed. All additional logic is placed in the user’s wallet and the verifier.

% The wallet is expected to operate in two stages: an offline \emph{Prepare} step, which verifies the Issuer’s signature once using standard libraries, parses and normalizes credential attributes (for example, turning a date of birth into an integer age), and commits to those attributes using a binding and hiding commitment scheme (a cryptographic way to lock values so they can later be revealed or proven in restricted form); and an online \emph{Show} step, which runs per presentation, where the wallet selects only the attributes or predicates required by a \emph{Relying Party}’s policy, proves them in zero knowledge against the stored commitments, and includes a fresh device signature over the session challenge to ensure the proof is tied to the holder’s device.

% A further requirement is \emph{modularity}: each major function, issuer signature verification, attribute commitment, predicate proofs, and device binding, should be defined as a separate module with a clear interface. This separation makes it possible to swap the underlying proof engine (for example, using a SNARK today or a post-quantum proof system in the future) without requiring changes to parts of the system that are costly or impractical to modify. The purpose of this modular view is to act as a comparison framework: it outlines how a deployment-friendly anonymous-credential stack could be structured, making it easier to compare proposals by the modules they cover, the constraints they address, and the trade-offs they make. 

% A summary of them with respect to the AC Framework is presented in Table~\ref{tab:rw-snapshot}.

% \paragraph{BBS-based anonymous credentials.~\cite{baum2024cryptographers}}
% BBS-based anonymous credentials are recommended in public feedback for the EUDI wallet as a way to meet the program’s requirement that presentations must not be tracked, linked, or correlated~\cite{baum2024cryptographers}.
% This work treats a credential as a constant-size signature on a vector of attributes in pairing-friendly groups, as introduced by Boneh–Boyen–Shacham and proven secure for BBS+ by Au–Susilo–Mu~\cite{C:BonBoySha04,SCN:AuSusMu06}.
% A holder then produces zero-knowledge proofs that reveal only the required attributes or predicates; each presentation is freshly generated so separate verifications cannot be linked.
% This matches our reference system view on the presentation side-privacy enforced at the holder with per-session, non-repeating outputs.
% Where these designs differ from our constraints is issuance. To use BBS/BBS+, issuers sign credentials with a pairing-based scheme rather than the RSA or ECDSA schemes used today~\cite{C:BonBoySha04,SCN:AuSusMu06}. To remain compatible with standardized curves such as P-256 while keeping public verifiability, a pairing-free, server-aided variant (often termed BBS\#) allows the holder to prefetch small auxiliary data through an oblivious interaction with an issuer-side helper and later perform non-interactive presentations; the helper data scales linearly with the number of planned presentations~\cite{cryptoeprint:2025/513}.
% In both variants, device binding and revocation checks can be encoded as attributes or verified within the proof so that transcripts and status queries avoid stable identifiers.


% \paragraph{Anonymous Credentials from ECDSA.~\cite{cryptoeprint:2024/2010}}
% This work considers environments where credential issuers already sign with ECDSA on standardized curves (such as P-256) and hash data with SHA-256.
% The main challenge is that proving correctness of an ECDSA signature in zero knowledge is costly with standard proof systems, because the arithmetic used in P-256 and the bit-level operations in SHA-256 do not align well with the fast polynomial techniques (such as number-theoretic transforms, a method that speeds up polynomial multiplication over special fields) that many modern ZK libraries rely on.
% To handle this, the authors introduce custom circuits for ECDSA and SHA-256, and use a layered protocol based on the sum-check technique with a lightweight encoding (Reed–Solomon code) to control proof size.
% An additional “consistency check” ensures that the same hidden signing key is used across both the signature and the hash logic.
% At presentation, the wallet produces a proof for the verifier and the device also signs a fresh challenge (this is the device-binding step: a live signature that ties the proof to the holder’s device).
% In terms of the reference system view, issuer compatibility is preserved, selective disclosure is supported, and device binding is included; however, there is no reusable offline phase, so the full proof is generated at every presentation. The reported costs are about 60\,ms to prove one ECDSA signature and about 1.2\,s for a complete mDL presentation on mobile devices~\cite[\S5.3,\S6.2]{cryptoeprint:2024/2010}, with larger proof sizes and higher verifier effort than systems based on succinct setup-dependent SNARKs.

% \paragraph{Crescent Credentials.~\cite{cryptoeprint:2024/2013}}
% This work considers environments where issuers continue using existing credential formats such as JWT or mDL and their current signing keys, so no issuer-side changes are required.
% Its workflow is split into a heavy one-time Prepare phase and a lightweight per-presentation Show phase.
% In Prepare, the wallet verifies the issuer’s signature, parses the credential into attributes, and creates two reusable artifacts, that is, cryptographic objects the wallet reuses across presentations: (i) a Groth16 proof that these checks were done correctly, and (ii) a Pedersen vector commitment over the attributes, enabling selective disclosure.
% Both artifacts support re-randomization for unlinkability.
% In the Show phase, the wallet re-randomizes the prepared artifacts and attaches only the proofs required by the verifier’s policy, such as proving an age threshold or linking two credentials to the same holder. Device binding can be added at this step by letting the secure element sign the verifier’s challenge.
% In terms of the reference system view, Crescent realizes the two-phase design with reusable offline work and modular predicates, while leaving issuers unchanged. The trade-offs are significant: the Prepare phase is heavy (tens of seconds for JWTs and minutes for mDLs), the scheme depends on pairing-based Groth16 proofs with a large trusted setup ($\approx$ 661 MB-1.1 GB~\cite[\S4]{cryptoeprint:2024/2013}), and the security model is classical only, without post-quantum protection. The Show step, however, runs with low latency-typically 22-41\,ms with $\approx$1 KB proofs, or about 315\,ms with device binding~\cite[\S4]{cryptoeprint:2024/2013}.

% \begin{table}[t]
% \centering
% \caption{Comparison of related approaches.}
% \label{tab:rw-snapshot}
% \footnotesize
% \begin{tabularx}{\linewidth}{l >{\RaggedRight\arraybackslash}X >{\RaggedRight\arraybackslash}X >{\RaggedRight\arraybackslash}X >{\RaggedRight\arraybackslash}X}
% \toprule
% \textbf{Feature} & \textbf{BBS/BBS+} & \textbf{AC from ECDSA} & \textbf{Crescent} & \textbf{zkID (ours)} \\
% \midrule
% Issuer modification & Required & None & None & None \\
% Offline phase       & None & None & Heavy Prepare; light Show & Lightweight, reusable Prepare \\
% Setup               & Pairing-based (no trusted setup) & Transparent & Large, per-circuit trusted setup & Transparent \\
% Proof mechanism     & Pairing-based signatures with ZK proofs & Sum-check with Ligero; custom ECDSA/SHA-256 circuits & Groth16 with Pedersen vector commitments; re-randomizable artifacts & Transparent sum-check; Hyrax-style vector commitments \\
% Device binding      & Optional & Included & Optional & Integrated (in-circuit) \\
% Reusability         & No & No & Yes & Yes \\
% \bottomrule
% \end{tabularx}
% \end{table}


% \subsection{Our zkID}

% \kwnote{A table or figure that summarizes the pros and cons of other approaches and ours will be very helpful for readers.}

% Our construction works with standardized credentials (e.g., SD-JWT, mDL) and existing PKI (RSA/ECDSA), so issuers do not need to change their issuance pipelines.
% The zkID workflow follows the two-phase split in the reference view: a one-time Prepare phase and a per-presentation Show phase.
% In Prepare, the wallet verifies the issuer’s signature, parses the credential into normalized messages, computes the associated hashes, and produces two reusable artifacts: (i) zero-knowledge proofs that issuer-side checks and parsing were done correctly, and (ii) Hyrax-style Pedersen vector commitments to a designated message column, supporting efficient proofs over multiple attributes.
% In Show, the wallet proves only the verifier’s requested predicates and includes a fresh device-binding signature. To link Prepare and Show without revealing values, the verifier checks equality of commitments across both proofs; the wallet reuses the corresponding randomness for that session.
% The proving backend is transparent (no trusted setup). It checks the arithmetic constraints with a sum-check–style protocol and uses a small inner-product check to verify commitment openings. For device binding, we choose a curve whose scalar field matches the device’s signature field (e.g., P-256), so the device signature can be verified directly inside the proof without emulation or field translation. 
% In terms of the reference system view, issuer compatibility is preserved, the two-phase reuse is integrated into the workflow, predicates are modular, and there is no trusted setup. The trade-offs are that security currently relies on discrete-log assumptions (not post-quantum) and that commitment equality requires using the same curve across Prepare and Show; the modular interface leaves room to swap in lattice-based commitments when suitable.