%ChatGPT

\noindent\textbf{Digital Identity.} A digital identity system typically consists of three roles: an \emph{issuer}, a \emph{holder}, and a 
\emph{verifier}. In the W3C Verifiable Credential Data Model, the issuer is a trusted authority responsible for  asserting \href{https://www.w3.org/TR/vc-data-model-1.1/\#dfn-claims}{claims} about one or more \href{https://www.w3.org/TR/vc-data-model-1.1/\#dfn-subjects}{subjects}, and creating a \href{https://www.w3.org/TR/vc-data-model-1.1/\#dfn-verifiable-credentials}{verifiable credential} from these \href{https://www.w3.org/TR/vc-data-model-1.1/\#dfn-claims}{claims}. The claims are digitally 
signed with the issuer’s private key. The signed credential is delivered securely to the user and stored in the wallet instance.

The holder retains these signed verifiable credentials and, during a presentation request, provides consent based on the verifier’s requirements. The holder sends the verifiable presentation (or proofs derived from them) to the verifier. The verifier, using the issuer’s public 
key, checks the digital signature to confirm authenticity and integrity of the received presentation, 
and may additionally verify validity period, or revocation status. This issuer-holder-
verifier pattern defines the standard operational model.

Such a system is illustrated in Fig.~\ref{fig:did_sys}.
% \begin{figure}[!h]
%     \centering
%     \resizebox{15cm}{!}{
%         \begin{tikzpicture}[
%           entity/.style = {draw, thick, rounded corners, minimum width=3.2cm, minimum height=1.2cm, align=center, fill=white},
%           edge/.style   = {->, >=Stealth, thick},
%           dashededge/.style = {->, >=Stealth, thick, dashed},
%           smalltext/.style = {font=\small, align=center}
%           ]
        
%           % Entities
%           \node[entity] (issuer) {Issuer\\(e.g., government, bank)};
%           \node[entity, right=4.8cm of issuer] (holder) {Holder / User\\(stores credential)};
%           \node[entity, right=4.8cm of holder] (verifier) {Verifier\\(relying party)};
        
%           % Issuance arrow: Issuer -> Holder
%           \draw[edge] (issuer.east) to[bend left=12] node[midway, above, smalltext] {Signed attribute set\\(credential)} (holder.west);
        
%           % Presentation arrow: Holder -> Verifier
%           \draw[edge] (holder.east) to[bend left=12] node[midway, above, smalltext] {Select attributes or proof\\presented to verifier} (verifier.west);
        
%           % Verification arrow: Verifier checks signature using issuer public key
%           \draw[edge] (verifier.west) to[bend left=25] node[midway, below, smalltext] {Verify signature \\(using issuer pk)} (issuer.east);
        
%           % Local storage box around holder to indicate secure storage (e.g., mobile wallet)
%           \node[draw=black, rounded corners, inner sep=6pt, fit=(holder), label=below:Secure storage (wallet)] {};
        
%           % Optional: Revocation / freshness check
%           \node[entity, below=3.0cm of holder] (revocation) {Revocation / Status Service};
%           \draw[dashededge] (verifier.south) to[bend left=12] node[midway, right, smalltext] {Check revocation \\/ freshness} (revocation.north);
%           \draw[dashededge] (revocation.north) to[bend left=12] node[midway, right, smalltext] {Respond\\ status} (verifier.south);
        
%           % Notes
%           \node[align=left, font=\footnotesize, below=0.6cm of revocation] {
%             \textbf{Notes:}\\
%             \begin{minipage}{20cm}
%               \vspace{1mm}
%               \begin{itemize}\itemsep1pt
%                 \item The issuer signs a set of user attributes and delivers the credential to the holder.
%                 \item The holder stores the signed attributes and selectively presents either the raw attributes or a derived proof to the verifier.
%                 \item The verifier validates authenticity by checking the issuer's signature and may consult a revocation/status service.
%               \end{itemize}
%             \end{minipage}
%           };
        
%         \end{tikzpicture}
%     }
%     \caption{Digital identity system}
%     \label{fig:did_sys}
% \end{figure}
\begin{figure}[!h]
  \centering
  \resizebox{0.9\linewidth}{!}{%
  \begin{tikzpicture}[
    entity/.style = {
      draw, thick, rounded corners,
      minimum height=1.1cm,
      align=center,
      inner sep=4pt,
      fill=white
    },
    edge/.style      = {->, >=stealth, thick, rounded corners=8pt},
    smalltext/.style = {font=\small, align=center}
  ]

    \node[entity, text width=3.8cm] (issuer)
      {Issuer\\\footnotesize(issues VCs)};
    \node[entity, right=4.8cm of issuer, text width=4.2cm] (holder)
      {Holder / User\\\footnotesize(stores VCs, creates VPs)};
    \node[entity, right=4.8cm of holder, text width=3.8cm] (verifier)
      {Verifier\\\footnotesize(verifies VPs)};

    \node[draw=black, rounded corners, inner sep=6pt,
          fit=(holder),
          label=above:Secure storage (wallet)] {};

    \node[entity, below=3.0cm of holder, text width=8.0cm] (registry)
      {Verifiable Data Registry\\
       \footnotesize(identifiers, schemas)};

    \draw[edge]
      (issuer.east) --
      node[pos=0.5, above, smalltext] {Issue credential (VC)}
      (holder.west);

    \draw[edge]
      (holder.east) --
      node[pos=0.5, above, smalltext] {Presentation (VP)}
      (verifier.west);

    \draw[edge]
      (issuer.south) |- 
      node[pos=0.25, left, smalltext]
        {Publish schemas}
      (registry.west);

    \draw[edge]
      (holder.south) --
      node[midway, right, smalltext]
        {Register identifiers\\and use schemas}
      (registry.north);

    \draw[edge]
      (verifier.south) |- 
      node[pos=0.25, right, smalltext]
        {Query\\identifiers \& schemas}
      (registry.east);

    % Notes
    \node[align=left, font=\footnotesize, below=0.6cm of registry] {
      \textbf{Notes:}\\
      \begin{minipage}{20cm}
        \vspace{1mm}
        \begin{itemize}\itemsep1pt
          \item The issuer issues a verifiable credential (VC) containing a set of user attributes and delivers it to the holder.
          \item The holder stores VCs in a secure wallet and selectively discloses attributes or derived proofs as a verifiable presentation (VP) to the verifier.
          \item The verifier validates authenticity by checking the issuer's signature using the issuer's public key (resolved via the verifiable data registry) and consults the registry to resolve identifiers and schemas.
        \end{itemize}
      \end{minipage}
    };

  \end{tikzpicture}%
  }
  \caption{Digital identity system}
  \label{fig:did_sys}
\end{figure}


Examples of such digital identity frameworks and credential formats that can be potentially compatible with OpenAC include:
\begin{itemize}
  \item \textbf{eIDAS / EUDI Architecture and Reference Framework} -European digital identity systems in which 
  qualified trust service providers or public authorities issue digitally signed identity 
  attributes that citizens can present via the EUDI Wallet.

  \item \textbf{Mobile Driver’s Licenses (mDL, ISO~18013-5)} - Government agencies issue a 
  digitally signed set of driving-related attributes (e.g., name, age, license class), which 
  users present to relying parties such as law enforcement or age-restricted service providers.

  \item \textbf{Verifiable Credentials (W3C VCDM)} -A general-purpose model where an issuer 
  signs JSON-LD or SD-JWT-based claims about a subject; holders store these credentials and 
  present them to verifiers who validate the issuer’s signature.

  % \item \textbf{Bank-Issued Digital Identity Schemes} --- Financial institutions validate 
  % customer identity information (e.g., KYC data) and issue signed digital credentials that can 
  % be reused for onboarding in other financial or commercial services.

  % \item \textbf{University or Organizational ID Systems} --- Institutions issue digitally 
  % signed attestations (e.g., enrollment status, roles, or qualifications) that students or 
  % employees can present to external verifiers for access to services or discounts.

  % \item \textbf{National Digital ID Cards (e.g., Estonia eID)} --- Government-issued smartcards 
  % or mobile IDs contain signed identity attributes that citizens present for authentication and 
  % authorization in public and private sector services.
\end{itemize}
