\noindent\textbf{BBS-based anonymous credentials}~\cite{baum2024cryptographers} have been recommended in public feedback for the EUDI Wallet as a 
means to ensure that presentations cannot be tracked, linked, or correlated~\cite{baum2024cryptographers}. 
These systems treat a credential as a constant-size signature on an attribute vector in pairing-friendly 
groups, following Boneh--Boyen--Shacham and the BBS$+$ security proofs of Au--Susilo--Mu%
~\cite{C:BonBoySha04,SCN:AuSusMu06}. Holders generate zero-knowledge proofs that selectively disclose 
only required attributes or predicates, with each presentation freshly randomized to prevent linkability. 
This aligns with our reference model in which presentation-side privacy is enforced via per-session, 
non-repeating outputs.

However, issuance differs from our constraints. Using BBS/BBS$+$ requires issuers to adopt a 
pairing-based signing scheme instead of the RSA or ECDSA mechanisms prevalent in existing ID systems. 
To retain compatibility with standardized curves such as P-256, a pairing-free server-aided variant 
(BBS\#) enables holders to obtain small auxiliary data through an oblivious interaction with an 
issuer-side helper and later produce non-interactive presentations; the helper data grows linearly 
with the number of planned presentations~\cite{cryptoeprint:2025/513}. Device binding and revocation 
can be encoded as attributes or verified inside the proof, ensuring transcripts and status queries 
do not introduce stable identifiers.
